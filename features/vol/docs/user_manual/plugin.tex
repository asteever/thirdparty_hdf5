\section{Creating a VOL Plugin}
Each VOL plugin should be of type {\tt H5VL\_class\_t} that is defined
as:

\begin{lstlisting}
/* Class information for each VOL driver */
typedef struct H5VL_class_t {
    H5VL_class_value_t value;
    const char *name;
    herr_t  (*initialize)(void);
    herr_t  (*terminate)(void);
    size_t  info_size;
    void *  (*fapl_copy)(const void *info);
    herr_t  (*fapl_free)(void *info);
    H5VL_attr_class_t          attr_cls;
    H5VL_datatype_class_t      datatype_cls;
    H5VL_dataset_class_t       dataset_cls;
    H5VL_file_class_t          file_cls;
    H5VL_group_class_t         group_cls;
    H5VL_link_class_t          link_cls;
    H5VL_object_class_t        object_cls;
    H5VL_async_class_t         async_cls;
} H5VL_class_t;
\end{lstlisting}

The {\tt value} field is an integer enum identifier that should be
greater than 128 for external plugins and smaller than 128 for
internal plugins. This plugin identifier is used to select the VOL
plugin to be used when creating/accessing the HDF5 container in the
application. Setting it in the VOL structure is required.

The {\tt name} field is a string that identifies the VOL plugin
name. Setting it is not required.

The {\tt initialize} field is a function pointer - MSC not used now!.

The {\tt terminate} field is a function pointer - MSC not used now!.

The {\tt info\_size} field indicates the size required to store the
info data that the plugin needs. That info data is passed when the
plugin is selected for usage with the file access property list (fapl)
function. It might be that the plugin defined does not require any
information from the user, which means the size in this field will be
zero. More information about the info data and the fapl selection
routines follow later.

The {\tt fapl\_copy} field is a function pointer that is called when
the plugin is selected with the fapl function. It allows the plugin to
make a copy if the info data since the user might free it when closing
the fapl. It is required if there is info data needed by the plugin.

The {\tt fapl\_free} field is a function pointer that is called to
free the info data when the fapl close routine is called. It is
required if there is info data needed by the plugin.

The rest of the fields in the {\tt H5VL\_class\_t} struct are
``subclasses'' that define all the object VOL function callbacks that
are mapped to from the HDF5 API layer and will be detailed in the
following sub-sections.

\subsection{Mapping the API to the Callbacks}
\label{sec:map}

The callback interface defined for the VOL has to be general enough to
handle all the HDF5 API operations that would access the
file. Furthermore it has to capture future additions to the HDF5
library with little to no changes to the callback interface. Changing
the interface often whenever new features are added would be
discouraging to plugin developers since that would mean reworking
their VOL plugin structure. To remedy this issue, every callback will
contain two parameters:
\begin{itemize}
\item A data transfer property list (DXPL) which allows that API to
  put some properties on for the plugins to retrieve if they have to
  for particular operations, without having to add arguments to the
  VOL callback function.
\item A pointer to a request ({\tt void **req}) to handle asynchronous
  operations if the HDF5 library adds support for them in future
  releases (beyond the 1.8 series). That pointer is set by the VOL
  plugin to a request object it creates to manage progress on that
  asynchronous operation. If the {\tt req} is {\tt NULL}, that means
  that the API operation is blocking and so the plugin would not
  execute the operation asynchronously. If the plugin does not support
  asynchronous operations, it needs not to worry about this field and
  leaves it unset.
\end{itemize}

In order to keep the number of the VOL object classes and callbacks
concise and readable, it was decided to not have a one-to-one mapping
between API operation and callbacks. Furthermore, to keep the
callbacks themselves short and not cluttered with a lot of parameters,
some of the parameters are passed in as properties in property lists
included with the callback. The value of those properties can be
retrieved by calling the public routine (or its private version if
this is an internal plugin): 
\begin{lstlisting}
herr_t H5Pget(hid_t plist_id, const char *property_name, void *value);
\end{lstlisting}
The property names and value types will be detailed when describing
each callback in their respective sections.

The HDF5 library provides several routines to access an object in the
container. For example to open an attribute on a group object, the
user could use {\tt H5Aopen()} and pass the group identifier directly
where the attribute needs to be opened. Alternatively, the user could
use {\tt H5Aopen\_by\_name()} or {\tt H5Aopen\_by\_idx()} to open the
attribute, which provides a more flexible way of locating the
attribute, whether by a starting object location and a path or an
index type and traversal order. All those types of accesses usually
map to one VOL callback with a parameter that indicates the access
type. In the example of opening an attribute, the three API open
routine will map to the same VOL open callback but with a different
location parameter. The same applies to all types of routines that
have multiple types of accesses.  The location parameter is a
structure defined as follows:

\begin{lstlisting}
/* 
 * Structure to hold parameters for object locations.
 * either: BY_ID, BY_NAME, BY_IDX, BY_ADDR, BY_REF 
 */

typedef struct H5VL_loc_params_t {
    H5I_type_t obj_type; /* The object type of the location object */
    H5VL_loc_type_t type; /* The location type */
    union { /* parameters of the location */
        struct H5VL_loc_by_name loc_by_name;
        struct H5VL_loc_by_idx  loc_by_idx;
        struct H5VL_loc_by_addr loc_by_addr;
        struct H5VL_loc_by_ref  loc_by_ref;
    }loc_data;
} H5VL_loc_params_t

/* 
 * Types for different ways that objects are located in an 
 * HDF5 container.
 */
typedef enum H5VL_loc_type_t {
    /* starting location is the target object*/
    H5VL_OBJECT_BY_SELF = 0, 

    /* location defined by object and path in H5VL_loc_by_name */
    H5VL_OBJECT_BY_NAME, 

    /* location defined by object, path, and index in H5VL_loc_by_idx */
    H5VL_OBJECT_BY_IDX,

    /* location defined by physical address in H5VL_loc_by_addr */
    H5VL_OBJECT_BY_ADDR,

    /* NOT USED */
    H5VL_OBJECT_BY_REF
} H5VL_loc_type_t;

struct H5VL_loc_by_name {
    const char *name; /* The path relative to the starting location */
    hid_t plist_id; /* The link access property list */
};

struct H5VL_loc_by_idx {
    const char *name; /* The path relative to the starting location */
    H5_index_t idx_type; /* Type of index */
    H5_iter_order_t order; /* Index traversal order */
    hsize_t n; /* position in index */
    hid_t plist_id; /* The link access property list */
};

struct H5VL_loc_by_addr {
    haddr_t addr; /* physical address of location */
};

/* Not used for now */
struct H5VL_loc_by_ref {
    H5R_type_t ref_type;
    const void *_ref;
    hid_t plist_id;
};
\end{lstlisting}

Another large set of operations that would make a one-to-one mapping
difficult are the {\tt Get} operations that retrieve something from an
object; for example a property list or a datatype of a dataset,
etc... To handle that, each class of objects has a general get
callback with a {\tt get\_type} and a {\tt va\_list} argument to handle
the multiple get operations. More information about types and the
arguments for each type will be detailed in the corresponding class
arguments.

Finally there are a set of functions for the file and general object
(H5O) classes that are not widely used or interesting enough for
plugin developers to implement. Those routines are mapped to a {\tt
  misc} callback in their respective class.

\subsection{The Attribute Function Callbacks}
The attribute API routines (H5A) allow HDF5 users to create and manage
HDF5 attributes. All the H5A API routines that modify the HDF5
container map to one of the attribute callback routines in this
class that the plugin needs to implement:

\begin{lstlisting}
typedef struct H5VL_attr_class_t {
    void *(*create)(void *obj, H5VL_loc_params_t loc_params, 
        const char *attr_name, hid_t acpl_id, hid_t aapl_id, 
        hid_t dxpl_id, void **req);

    void *(*open)(void *obj, H5VL_loc_params_t loc_params, 
        const char *attr_name, hid_t aapl_id, hid_t dxpl_id, void **req);

    herr_t (*read)(void *attr, hid_t mem_type_id, void *buf, 
        hid_t dxpl_id, void **req);

    herr_t (*write)(void *attr, hid_t mem_type_id, const void *buf, 
        hid_t dxpl_id, void **req);

    herr_t (*iterate)(void *obj, H5VL_loc_params_t loc_params,
        H5_index_t idx_type, H5_iter_order_t order, hsize_t *n, 
        H5A_operator2_t  op, void *op_data, hid_t dxpl_id, void **req);

    herr_t (*get)(void *obj, H5VL_attr_get_t get_type, hid_t dxpl_id, 
        void **req, va_list arguments);

    herr_t (*remove)(void *obj, H5VL_loc_params_t loc_params, 
        const char *attr_name, hid_t dxpl_id, void **req);

    herr_t (*close)(void *attr, hid_t dxpl_id, void **req);
} H5VL_attr_class_t;
\end{lstlisting}

The {\tt create} callback in the attribute class should create an
attribute object in the container of the location object and
returns a pointer to the attribute structure containing information to
access the attribute in future calls. 

\textbf{Signature:}
\begin{lstlisting}
    void *(*create)(void *obj, H5VL_loc_params_t loc_params, 
        const char *attr_name, hid_t acpl_id, hid_t aapl_id, 
        hid_t dxpl_id, void **req);
\end{lstlisting}

\textbf{Arguments:}\\
\begin{tabular}{l p{10cm}}
  {\tt obj} & (IN): Pointer to an object where the attribute needs
  to be created or where the look-up of the target object needs to
  start.\\
  {\tt loc\_params} & (IN): The location parameters as explained in
  section~\ref{sec:map}.\\
  {\tt attr\_name} & (IN): The name of the attribute to be created.\\
  {\tt acpl\_id} & (IN): The attribute creation property list. It contains
  all the attribute creation properties in addition to the attribute
  datatype (an {\tt hid\_t}) and dataspace (an {\tt hid\_t}) that can
  be retrieved with the properties, {\tt H5VL\_ATTR\_TYPE\_ID} and
  {\tt H5VL\_ATTR\_SPACE\_ID}.\\
  {\tt aapl\_id} & (IN): The attribute access property list.\\
  {\tt dxpl\_id} & (IN): The data transfer property list.\\
  {\tt req} & (IN/OUT): A pointer to the asynchronous request of the
  operation created by the plugin.\\
\end{tabular}

The {\tt open} callback in the attribute class should open an
attribute object in the container of the location object and returns a
pointer to the attribute structure containing information to access
the attribute in future calls. 

\textbf{Signature:}
\begin{lstlisting}
    void *(*open)(void *obj, H5VL_loc_params_t loc_params, 
        const char *attr_name, hid_t aapl_id, hid_t dxpl_id, void **req);
\end{lstlisting}

\textbf{Arguments:}\\
\begin{tabular}{l p{10cm}}
  {\tt obj} & (IN): Pointer to an object where the attribute needs to be
  opened or where the look-up of the target object needs to start.\\
  {\tt loc\_params} & (IN): The location parameters as explained in
  section~\ref{sec:map}.\\
  {\tt attr\_name} & (IN): The name of the attribute to be opened.\\
  {\tt aapl\_id} & (IN): The attribute access property list.\\
  {\tt dxpl\_id} & (IN): The data transfer property list.\\
  {\tt req} & (IN/OUT): A pointer to the asynchronous request of the
  operation created by the plugin.\\
\end{tabular}

The {\tt read} callback in the attribute class should read data from
the attribute object and returns an {\tt herr\_t} indicating success or
failure.

\textbf{Signature:}
\begin{lstlisting}
    herr_t (*read)(void *attr, hid_t mem_type_id, void *buf, 
        hid_t dxpl_id, void **req);
\end{lstlisting}

\textbf{Arguments:}\\
\begin{tabular}{l p{10cm}}
  {\tt attr} & (IN): Pointer to the attribute object.\\
  {\tt mem\_type\_id} & (IN): The memory datatype of the attribute.\\
  {\tt buf} & (OUT): Data buffer to be read into.\\
  {\tt dxpl\_id} & (IN): The data transfer property list.\\
  {\tt req} & (IN/OUT): A pointer to the asynchronous request of the
  operation created by the plugin.\\
\end{tabular}

The {\tt write} callback in the attribute class should write data to
the attribute object and returns an {\tt herr\_t} indicating success or
failure.

\textbf{Signature:}
\begin{lstlisting}
    herr_t (*write)(void *attr, hid_t mem_type_id, const void *buf, 
        hid_t dxpl_id, void **req);
\end{lstlisting}

\textbf{Arguments:}\\
\begin{tabular}{l p{10cm}}
  {\tt attr} & (IN): Pointer to the attribute object.\\
  {\tt mem\_type\_id} & (IN): The memory datatype of the attribute.\\
  {\tt buf} & (IN): Data buffer to be written.\\
  {\tt dxpl\_id} & (IN): The data transfer property list.\\
  {\tt req} & (IN/OUT): A pointer to the asynchronous request of the
  operation created by the plugin.\\
\end{tabular}

The {\tt iterate} callback in the attribute class should iterate over
the attributes in the container of the location object and call the
user defined function on each one. It returns an {\tt herr\_t}
indicating success or failure.

\textbf{Signature:}
\begin{lstlisting}
    herr_t (*iterate)(void *obj, H5VL_loc_params_t loc_params,
        H5_index_t idx_type, H5_iter_order_t order, hsize_t *n, 
        H5A_operator2_t  op, void *op_data, hid_t dxpl_id, void **req);
\end{lstlisting}

\textbf{Arguments:}\\
\begin{tabular}{l p{10cm}}
  {\tt obj} & (IN): Pointer to an object where the iteration needs
  to happen or where the look-up of the target object needs to
  start.\\
  {\tt loc\_params} & (IN): The location parameters as
  explained in section~\ref{sec:map}.\\
  {\tt idx\_type} & (IN): Type of index.\\
  {\tt order} & (IN): Order in which to iterate over index.\\
  {\tt n} & (IN/OUT): Initial and return offset withing index.\\
  {\tt op} & (IN): User-defined function to pass each
  attribute to. \\
  {\tt op\_data} & (IN/OUT): User data to pass through to and to be
  returned by iterator operator function. \\
  {\tt dxpl\_id} & (IN): The data transfer property list.\\
  {\tt req} & (IN/OUT): A pointer to the asynchronous request of the
  operation created by the plugin.\\
\end{tabular}

The {\tt get} callback in the attribute class should retrieve
information about the attribute as specified in the {\tt get\_type}
parameter.It returns an {\tt herr\_t} indicating success or failure.

\textbf{Signature:}
\begin{lstlisting}
    herr_t (*get)(void *obj, H5VL_attr_get_t get_type, hid_t dxpl_id, 
        void **req, va_list arguments);
\end{lstlisting}

The {\tt get\_type} argument is an {\tt enum}:
\begin{lstlisting}
/* types for all attr get API routines */
typedef enum H5VL_attr_get_t {
    H5VL_ATTR_EXISTS = 0,       /* attribute exists?       */
    H5VL_ATTR_GET_SPACE,        /* dataspace               */
    H5VL_ATTR_GET_TYPE,         /* datatype                */
    H5VL_ATTR_GET_ACPL,         /* creation property list  */
    H5VL_ATTR_GET_NAME,         /* access property list    */
    H5VL_ATTR_GET_STORAGE_SIZE, /* storage size            */
    H5VL_ATTR_GET_INFO          /* offset                  */
} H5VL_attr_get_t;
\end{lstlisting}

\textbf{Arguments:}\\
\begin{tabular}{l p{10cm}}
  {\tt attr} & (IN): An attribute or location object where information
  needs to be retrieved from.\\
  {\tt get\_type} & (IN): The type of the information to retrieve.\\
  {\tt dxpl\_id} & (IN): The data transfer property list.\\
  {\tt req} & (IN/OUT): A pointer to the asynchronous request of the
  operation created by the plugin.\\
  {\tt arguments} & (IN/OUT): argument list containing parameters and
  output pointers for the get operation. \\
\end{tabular}

The {\tt arguments} argument contains a variable list of arguments
depending on the {\tt get\_type} parameter. The following list shows
the argument list, in-order, for each type:

\begin{itemize}
\item {\tt H5VL\_ATTR\_EXISTS}, to check if an attribute exists on a
  particular object specified in {\tt obj}:
  \begin{enumerate}
  \item {\tt H5VL\_loc\_params\_t loc\_params} (IN): The location parameters
    explained in section~\ref{sec:map}.
  \item {\tt char *attr\_name} (IN): the attribute name to check.
  \item {\tt htri\_t *ret} (OUT): existence result, 0 if false, 1 if true.
  \end{enumerate}

\item {\tt H5VL\_ATTR\_GET\_SPACE}, to retrieve the dataspace of the
  attribute specified in {\tt obj}:
  \begin{enumerate}
  \item {\tt hid\_t *ret\_id} (OUT): a pointer to an identifier of the
    attribute dataspace.
  \end{enumerate}

\item {\tt H5VL\_ATTR\_GET\_TYPE}, to retrieve the datatype of the
  attribute specified in {\tt obj}:
  \begin{enumerate}
  \item {\tt hid\_t *ret\_id} (OUT): a pointer to an identifier of the
    attribute datatype.
  \end{enumerate}

\item {\tt H5VL\_ATTR\_GET\_ACPL}, to retrieve the attribute creation
  property list of the attribute specified in {\tt obj}:
  \begin{enumerate}
  \item {\tt hid\_t *ret\_id} (OUT): a pointer to an identifier of the
    attribute creation property list.
  \end{enumerate}

\item {\tt H5VL\_ATTR\_GET\_NAME}, to retrieve an attribute name on a
  particular object specified in {\tt obj}:
  \begin{enumerate}
  \item {\tt H5VL\_loc\_params\_t loc\_params} (IN): The location parameters
    explained in section~\ref{sec:map}. The type could be either
    {\tt H5VL\_OBJECT\_BY\_SELF} meaning {\tt obj} is the attribute,
    or {\tt H5VL\_OBJECT\_BY\_IDX} meaning the attribute to retrieve
    the name for should be looked up using the index information on
    the object in {\tt obj} and the index information in {\tt loc\_params}.
  \item {\tt size\_t buf\_size} (IN): the size of the buffer to store
    the name in.
  \item {\tt void *buf} (OUT): Buffer to store the name in.
  \item {\tt ssize\_t *ret\_val} (OUT): return the actual size needed
    to store the fill attribute name.
  \end{enumerate}

\item {\tt H5VL\_ATTR\_GET\_INFO}, to retrieve the attribute info:
  \begin{enumerate}
  \item {\tt H5VL\_loc\_params\_t loc\_params} (IN): The location parameters
    explained in section~\ref{sec:map}. 
  \item {\tt H5A\_info\_t *ainfo} (OUT): info structure to fill the
    attribute info in.
  \end{enumerate}

\item {\tt H5VL\_ATTR\_GET\_STORAGE\_SIZE}, to retrieve the storage
  size of the attribute specified in {\tt obj}:
  \begin{enumerate}
  \item {\tt hsize\_t *ret} (OUT): a pointer to the storage size of
    the attribute in the container.
  \end{enumerate}

\end{itemize}

The {\tt remove} callback in the attribute class should remove an
attribute object in the container of the location object and returns
an {\tt herr\_t} indicating success or failure.

\textbf{Signature:}
\begin{lstlisting}
    herr_t (*remove)(void *obj, H5VL_loc_params_t loc_params, 
        const char *attr_name, hid_t dxpl_id, void **req);
\end{lstlisting}

\textbf{Arguments:}\\
\begin{tabular}{l p{10cm}}
  {\tt obj} & (IN): Pointer to an object where the attribute needs
  to be removed or where the look-up of the target object needs to
  start.\\
  {\tt loc\_params} & (IN): The location parameters as explained in
  section~\ref{sec:map}.\\
  {\tt attr\_name} & (IN): The name of the attribute to be removed.\\
  {\tt dxpl\_id} & (IN): The data transfer property list.\\
  {\tt req} & (IN/OUT): A pointer to the asynchronous request of the
  operation created by the plugin.\\
\end{tabular}

The {\tt close} callback in the attribute class should terminate
access to the attribute object and free all resources it was
consuming, and returns an {\tt herr\_t} indicating success or failure.

\textbf{Signature:}
\begin{lstlisting}
    herr_t (*close)(void *attr, hid_t dxpl_id, void **req);
\end{lstlisting}

\textbf{Arguments:}\\
\begin{tabular}{l p{10cm}}
  {\tt attr} & (IN): Pointer to the attribute object.\\
  {\tt dxpl\_id} & (IN): The data transfer property list.\\
  {\tt req} & (IN/OUT): A pointer to the asynchronous request of the
  operation created by the plugin.\\
\end{tabular}

\subsection{The Named Datatype Function Callbacks}
The HDF5 datatype routines (H5T) allow users to create and manage HDF5
datatypes. Those routines are divided into two categories. One that
operates on all types of datatypes but do not modify the contents of
the container (all in memory), and others that operate on named
datatypes by accessing the container. When a user creates an HDF5
datatype, it is still an object in memory space (transient datatype)
that has not been added to the HDF5 containers. Only when a user
commits the HDF5 datatype, it becomes persistent in the
container. Those are called named/committed datatypes. The transient
H5T routines should work on named datatypes nevertheless. 

All the H5T API routines that modify the HDF5 container map to one of
the named datatype callback routines in this class that the plugin needs to
implement:

\begin{lstlisting}
typedef struct H5VL_datatype_class_t {
    void *(*commit)(void *obj, H5VL_loc_params_t loc_params, 
        const char *name, hid_t type_id, hid_t lcpl_id, hid_t tcpl_id, 
        hid_t tapl_id, hid_t dxpl_id, void **req);

    void *(*open) (void *obj, H5VL_loc_params_t loc_params, 
        const char * name, hid_t tapl_id, hid_t dxpl_id, void **req);

    ssize_t (*get_binary)(void *obj, unsigned char *buf, size_t size, 
        hid_t dxpl_id, void **req);

    herr_t (*get) (void *obj, H5VL_datatype_get_t get_type, 
        hid_t dxpl_id, void **req, va_list arguments);

    herr_t (*close) (void *dt, hid_t dxpl_id, void **req);
} H5VL_datatype_class_t;
\end{lstlisting}

The {\tt commit} callback in the named datatype class should create a datatype object in the container of the location object and
returns a pointer to the datatype structure containing information to
access the datatype in future calls. 

\textbf{Signature:}
\begin{lstlisting}
    void *(*commit)(void *obj, H5VL_loc_params_t loc_params, 
        const char *name, hid_t type_id, hid_t lcpl_id, hid_t tcpl_id, 
        hid_t tapl_id, hid_t dxpl_id, void **req);
\end{lstlisting}

\textbf{Arguments:}\\
\begin{tabular}{l p{10cm}}
  {\tt obj} & (IN): Pointer to an object where the datatype needs
  to be committed or where the look-up of the target object needs to
  start.\\
  {\tt loc\_params} & (IN): The location parameters as explained in
  section~\ref{sec:map}. In this call, the location type is always {\tt
    H5VL\_OBJECT\_BY\_SELF}. \\
  {\tt name} & (IN): The name of the datatype to be created.\\
  {\tt type\_id} & (IN): The transient datatype identifier to be
  committed. \\
  {\tt lcpl\_id} & (IN): The link creation property list. \\
  {\tt tcpl\_id} & (IN): The datatype creation property list.\\
  {\tt tapl\_id} & (IN): The datatype access property list.\\
  {\tt dxpl\_id} & (IN): The data transfer property list.\\
  {\tt req} & (IN/OUT): A pointer to the asynchronous request of the
  operation created by the plugin.\\
\end{tabular}

The {\tt open} callback in the named datatype class should open a
previously committed datatype object in the container of the location
object and returns a pointer to the datatype structure containing
information to access the datatype in future calls.

\textbf{Signature:}
\begin{lstlisting}
    void *(*open) (void *obj, H5VL_loc_params_t loc_params, 
        const char * name, hid_t tapl_id, hid_t dxpl_id, void **req);
\end{lstlisting}

\textbf{Arguments:}\\
\begin{tabular}{l p{10cm}}
  {\tt obj} & (IN): Pointer to an object where the datatype needs
  to be opened or where the look-up of the target object needs to
  start.\\
  {\tt loc\_params} & (IN): The location parameters as explained in
  section~\ref{sec:map}. In this call, the location type is always {\tt
    H5VL\_OBJECT\_BY\_SELF}. \\
  {\tt name} & (IN): The name of the datatype to be opened.\\
  {\tt tapl\_id} & (IN): The datatype access property list.\\
  {\tt dxpl\_id} & (IN): The data transfer property list.\\
  {\tt req} & (IN/OUT): A pointer to the asynchronous request of the
  operation created by the plugin.\\
\end{tabular}

The {\tt get\_binary} callback in the named datatype class should
serialize the original transient HDF5 datatype that was committed, or
return the size that is required for it be serialized if the passed in
buffer is {\tt NULL}. The HDF5 library provides two functions to
encode and decode datatypes in their transient form, {\tt H5Tencode()}
and {\tt H5Tdecode()}. When a datatype is committed, the plugin is
required to keep the serialized form of the transient datatype stored
somewhere in the container (which is usually the case anyway when
committing a named datatype), so it can be retrieved with this
call. This is needed to generate the higher level HDF5 datatype
identifier that allows all the H5T ``transient'' routines to work
properly on the named datatype.

\textbf{Signature:}
\begin{lstlisting}
    ssize_t (*get_binary)(void *obj, unsigned char *buf, size_t size, 
        hid_t dxpl_id, void **req);
\end{lstlisting}

\textbf{Arguments:}\\
\begin{tabular}{l p{10cm}}
  {\tt obj} & (IN): Pointer to the named datatype object.\\
  {\tt buf} & (OUT): Buffer to out the binary form of the datatype in.\\
  {\tt size} & (IN): The size of the buffer passed in (0 if NULL).\\
  {\tt dxpl\_id} & (IN): The data transfer property list.\\
  {\tt req} & (IN/OUT): A pointer to the asynchronous request of the
  operation created by the plugin.\\
\end{tabular}

The {\tt get} callback in the named datatype class should retrieve
information about the named datatype as specified in the {\tt get\_type}
parameter.It returns an {\tt herr\_t} indicating success or failure.

\textbf{Signature:}
\begin{lstlisting}
    herr_t (*get) (void *obj, H5VL_datatype_get_t get_type, 
        hid_t dxpl_id, void **req, va_list arguments);
\end{lstlisting}

The {\tt get\_type} argument is an {\tt enum}:
\begin{lstlisting}
/* types for all datatype get API routines */
typedef enum H5VL_datatype_get_t {
    H5VL_DATATYPE_GET_TCPL = 0 /*datatype creation property list */
} H5VL_datatype_get_t;
\end{lstlisting}

\textbf{Arguments:}\\
\begin{tabular}{l p{10cm}}
  {\tt obj} & (IN): The named datatype to retrieve information from.\\
  {\tt get\_type} & (IN): The type of the information to retrieve.\\
  {\tt dxpl\_id} & (IN): The data transfer property list.\\
  {\tt req} & (IN/OUT): A pointer to the asynchronous request of the
  operation created by the plugin.\\
  {\tt arguments} & (IN/OUT): argument list containing parameters and
  output pointers for the get operation. \\
\end{tabular}

The {\tt arguments} argument contains a variable list of arguments
depending on the {\tt get\_type} parameter. The following list shows
the argument list, in-order, for each type:

\begin{itemize}
\item {\tt H5VL\_DATATYPE\_GET\_TCPL}, to retrieve the datatype
  creation property list:
  \begin{enumerate}
  \item {\tt hid\_t *ret\_id} (OUT): a pointer to an identifier of the
    type creation property list.
  \end{enumerate}
\end{itemize}

The {\tt close} callback in the named datatype class should terminate
access to the datatype object and free all resources it was
consuming, and returns an {\tt herr\_t} indicating success or failure.

\textbf{Signature:}
\begin{lstlisting}
    herr_t (*close) (void *dt, hid_t dxpl_id, void **req);
\end{lstlisting}

\textbf{Arguments:}\\
\begin{tabular}{l p{10cm}}
  {\tt dt} & (IN): Pointer to the datatype object.\\
  {\tt dxpl\_id} & (IN): The data transfer property list.\\
  {\tt req} & (IN/OUT): A pointer to the asynchronous request of the
  operation created by the plugin.\\
\end{tabular}

\subsection{The Dataset Function Callbacks}

The dataset API routines (H5D) allow HDF5 users to create and manage
HDF5 datasets. All the H5D API routines that modify the HDF5 container
map to one of the dataset callback routines in this class that the
plugin needs to implement:

\begin{lstlisting}
typedef struct H5VL_dataset_class_t {
    void *(*create)(void *obj, H5VL_loc_params_t loc_params, 
        const char *name, hid_t dcpl_id, hid_t dapl_id, 
        hid_t dxpl_id, void **req);

    void *(*open)(void *obj, H5VL_loc_params_t loc_params, 
        const char *name, hid_t dapl_id, hid_t dxpl_id, void **req);

    herr_t (*read)(void *dset, hid_t mem_type_id, hid_t mem_space_id, 
        hid_t file_space_id, hid_t dxpl_id, void *buf, void **req);

    herr_t (*write)(void *dset, hid_t mem_type_id, hid_t mem_space_id, 
        hid_t file_space_id, hid_t dxpl_id, const void * buf, void **req);

    herr_t (*set_extent)(void *dset, const hsize_t size[], 
        hid_t dxpl_id, void **req);

    herr_t (*get)(void *dset, H5VL_dataset_get_t get_type, 
        hid_t dxpl_id, void **req, va_list arguments);

    herr_t (*close) (void *dset, hid_t dxpl_id, void **req);
} H5VL_dataset_class_t;
\end{lstlisting}

The {\tt create} callback in the dataset class should create a dataset
object in the container of the location object and returns a pointer
to the dataset structure containing information to access the dataset
in future calls.

\textbf{Signature:}
\begin{lstlisting}
    void *(*create)(void *obj, H5VL_loc_params_t loc_params, 
        const char *name, hid_t dcpl_id, hid_t dapl_id, 
        hid_t dxpl_id, void **req);
\end{lstlisting}

\textbf{Arguments:}\\
\begin{tabular}{l p{10cm}}
  {\tt obj} & (IN): Pointer to an object where the dataset needs
  to be created or where the look-up of the target object needs to
  start.\\
  {\tt loc\_params} & (IN): The location parameters as explained in
  section~\ref{sec:map}. The type can be only {\tt
    H5VL\_OBJECT\_BY\_SELF} in this callback. \\
  {\tt name} & (IN): The name of the dataset to be created.\\
  {\tt dcpl\_id} & (IN): The dataset creation property list. It contains
  all the dataset creation properties in addition to the dataset
  datatype (an {\tt hid\_t}), dataspace (an {\tt hid\_t}), and the
  link creation property list of the create operation (an {\tt
    hid\_t}) that can be retrieved with the properties, {\tt
    H5VL\_DSET\_TYPE\_ID}, {\tt H5VL\_DSET\_SPACE\_ID},  and {\tt
    H5VL\_DSET\_LCPL\_ID} respectively.\\
  {\tt dapl\_id} & (IN): The dataset access property list.\\
  {\tt dxpl\_id} & (IN): The data transfer property list.\\
  {\tt req} & (IN/OUT): A pointer to the asynchronous request of the
  operation created by the plugin.\\
\end{tabular}

The {\tt open} callback in the dataset class should open a dataset
object in the container of the location object and returns a pointer
to the dataset structure containing information to access the dataset
in future calls.

\textbf{Signature:}
\begin{lstlisting}
    void *(*open)(void *obj, H5VL_loc_params_t loc_params, 
        const char *name, hid_t dapl_id, hid_t dxpl_id, void **req);
\end{lstlisting}

\textbf{Arguments:}\\
\begin{tabular}{l p{10cm}}
  {\tt obj} & (IN): Pointer to an object where the dataset needs to be
  opened or where the look-up of the target object needs to start.\\
  {\tt loc\_params} & (IN): The location parameters as explained in
  section~\ref{sec:map}. The type can be only {\tt
    H5VL\_OBJECT\_BY\_SELF} in this callback. \\
  {\tt name} & (IN): The name of the dataset to be opened.\\
  {\tt dapl\_id} & (IN): The dataset access property list.\\
  {\tt dxpl\_id} & (IN): The data transfer property list.\\
  {\tt req} & (IN/OUT): A pointer to the asynchronous request of the
  operation created by the plugin.\\
\end{tabular}

The {\tt read} callback in the dataset class should read data from
the dataset object and returns an {\tt herr\_t} indicating success or
failure.

\textbf{Signature:}
\begin{lstlisting}
    herr_t (*read)(void *dset, hid_t mem_type_id, hid_t mem_space_id, 
        hid_t file_space_id, hid_t dxpl_id, void *buf, void **req);
\end{lstlisting}

\textbf{Arguments:}\\
\begin{tabular}{l p{10cm}}
  {\tt dset} & (IN): Pointer to the dataset object.\\
  {\tt mem\_type\_id} & (IN): The memory datatype of the data.\\
  {\tt mem\_space\_id} & (IN): The memory dataspace selection.\\
  {\tt file\_space\_id} & (IN): The file dataspace selection.\\
  {\tt dxpl\_id} & (IN): The data transfer property list.\\
  {\tt buf} & (OUT): Data buffer to be read into.\\
  {\tt req} & (IN/OUT): A pointer to the asynchronous request of the
  operation created by the plugin.\\
\end{tabular}

The {\tt write} callback in the dataset class should write data to
the dataset object and returns an {\tt herr\_t} indicating success or
failure.

\textbf{Signature:}
\begin{lstlisting}
    herr_t (*write)(void *dset, hid_t mem_type_id, hid_t mem_space_id, 
        hid_t file_space_id, hid_t dxpl_id, const void * buf, void **req);
\end{lstlisting}

\textbf{Arguments:}\\
\begin{tabular}{l p{10cm}}
  {\tt dset} & (IN): Pointer to the dataset object.\\
  {\tt mem\_type\_id} & (IN): The memory datatype of the data.\\
  {\tt mem\_space\_id} & (IN): The memory dataspace selection.\\
  {\tt file\_space\_id} & (IN): The file dataspace selection.\\
  {\tt dxpl\_id} & (IN): The data transfer property list.\\
  {\tt buf} & (IN): Data buffer to be written from.\\
  {\tt req} & (IN/OUT): A pointer to the asynchronous request of the
  operation created by the plugin.\\
\end{tabular}

The {\tt set\_extent} callback in the dataset class should extend the
dataset dimensions and returns an {\tt herr\_t} indicating success or
failure.

\textbf{Signature:}
\begin{lstlisting}
    herr_t (*set_extent)(void *dset, const hsize_t size[], 
        hid_t dxpl_id, void **req);
\end{lstlisting}

\textbf{Arguments:}\\
\begin{tabular}{l p{10cm}}
  {\tt dset} & (IN): Pointer to the dataset object.\\
  {\tt size} & (IN): new dimensions of the dataset.\\
  {\tt dxpl\_id} & (IN): The data transfer property list.\\
  {\tt req} & (IN/OUT): A pointer to the asynchronous request of the
  operation created by the plugin.\\
\end{tabular}

The {\tt get} callback in the dataset class should retrieve
information about the dataset as specified in the {\tt get\_type}
parameter.It returns an {\tt herr\_t} indicating success or failure.

\textbf{Signature:}
\begin{lstlisting}
    herr_t (*get)(void *dset, H5VL_dataset_get_t get_type, 
        hid_t dxpl_id, void **req, va_list arguments);
\end{lstlisting}

The {\tt get\_type} argument is an {\tt enum}:
\begin{lstlisting}
/* types for all dataset get API routines */
typedef enum H5VL_dataset_get_t {
    H5VL_DATASET_GET_SPACE = 0,     /* dataspace                */
    H5VL_DATASET_GET_SPACE_STATUS,  /* space status             */
    H5VL_DATASET_GET_TYPE,          /* datatype                 */
    H5VL_DATASET_GET_DCPL,          /* creation property list   */
    H5VL_DATASET_GET_DAPL,          /* access property list     */
    H5VL_DATASET_GET_STORAGE_SIZE,  /* storage size             */
    H5VL_DATASET_GET_OFFSET         /* offset                   */
} H5VL_dataset_get_t;
\end{lstlisting}

\textbf{Arguments:}\\
\begin{tabular}{l p{10cm}}
  {\tt dset} & (IN): The dataset object where information needs to be
  retrieved from.\\
  {\tt get\_type} & (IN): The type of the information to retrieve.\\
  {\tt dxpl\_id} & (IN): The data transfer property list.\\
  {\tt req} & (IN/OUT): A pointer to the asynchronous request of the
  operation created by the plugin.\\
  {\tt arguments} & (IN/OUT): argument list containing parameters and
  output pointers for the get operation. \\
\end{tabular}

The {\tt arguments} argument contains a variable list of arguments
depending on the {\tt get\_type} parameter. The following list shows
the argument list, in-order, for each type:

\begin{itemize}
\item {\tt H5VL\_DATASET\_GET\_SPACE}, to retrieve the dataspace of the
  dataset specified in {\tt obj}:
  \begin{enumerate}
  \item {\tt hid\_t *ret\_id} (OUT): a pointer to an identifier of the
    dataset dataspace.
  \end{enumerate}

\item {\tt H5VL\_DATASET\_GET\_SPACE\_STATUS}, to retrieve the
  information whether space has been allocated for the dataset:
  \begin{enumerate}
  \item {\tt H5D\_space\_status\_t *allocation} (OUT): a pointer to
    space status.
  \end{enumerate}

\item {\tt H5VL\_DATASET\_GET\_TYPE}, to retrieve the datatype of the
  dataset specified in {\tt obj}:
  \begin{enumerate}
  \item {\tt hid\_t *ret\_id} (OUT): a pointer to an identifier of the
    dataset datatype.
  \end{enumerate}

\item {\tt H5VL\_DATASET\_GET\_DCPL}, to retrieve the dataset creation
  property list of the dataset specified in {\tt obj}:
  \begin{enumerate}
  \item {\tt hid\_t *ret\_id} (OUT): a pointer to an identifier of the
    dataset creation property list.
  \end{enumerate}

\item {\tt H5VL\_DATASET\_GET\_DAPL}, to retrieve the dataset access
  property list of the dataset specified in {\tt obj}:
  \begin{enumerate}
  \item {\tt hid\_t *ret\_id} (OUT): a pointer to an identifier of the
    dataset access property list.
  \end{enumerate}

\item {\tt H5VL\_DATASET\_GET\_STORAGE\_SIZE}, to retrieve the storage
  size of the dataset specified in {\tt obj}:
  \begin{enumerate}
  \item {\tt hsize\_t *ret} (OUT): a pointer to the storage size of
    the dataset in the container.
  \end{enumerate}

\item {\tt H5VL\_DATASET\_GET\_OFFSET}, to retrieve the offset of the
  dataset specified in {\tt obj} in the container:
  \begin{enumerate}
  \item {\tt haddr\_t *ret} (OUT): a pointer to the offset of the
    dataset in the container.
  \end{enumerate}
\end{itemize}

The {\tt close} callback in the dataset class should terminate access
to the dataset object and free all resources it was consuming, and
returns an {\tt herr\_t} indicating success or failure.

\textbf{Signature:}
\begin{lstlisting}
    herr_t (*close)(void *dset, hid_t dxpl_id, void **req);
\end{lstlisting}

\textbf{Arguments:}\\
\begin{tabular}{l p{10cm}}
  {\tt dset} & (IN): Pointer to the dataset object.\\
  {\tt dxpl\_id} & (IN): The data transfer property list.\\
  {\tt req} & (IN/OUT): A pointer to the asynchronous request of the
  operation created by the plugin.\\
\end{tabular}

\subsection{The File Function Callbacks}

\begin{lstlisting}
typedef struct H5VL_file_class_t {
    void *(*create)(const char *name, unsigned flags, hid_t fcpl_id,
        hid_t fapl_id, hid_t dxpl_id, void **req);

    void *(*open)(const char *name, unsigned flags, hid_t fapl_id, 
        hid_t dxpl_id, void **req);

    herr_t (*flush)(void *obj, H5VL_loc_params_t loc_params, 
        H5F_scope_t scope, hid_t dxpl_id, void **req);

    herr_t (*get)(void *file, H5VL_file_get_t get_type, hid_t dxpl_id, 
        void **req, va_list arguments);

    herr_t (*misc)(void *file, H5VL_file_misc_t misc_type, 
        hid_t dxpl_id, void **req, va_list arguments);

    herr_t (*optional)(void *file, H5VL_file_optional_t op_type, 
        hid_t dxpl_id, void **req, va_list arguments);

    herr_t (*close) (void *file, hid_t dxpl_id, void **req);
} H5VL_file_class_t;
\end{lstlisting}

\subsection{The Group Function Callbacks}

The group API routines (H5G) allow HDF5 users to create and manage
HDF5 groups. All the H5G API routines that modify the HDF5 container
map to one of the group callback routines in this class that the
plugin needs to implement:

\begin{lstlisting}
typedef struct H5VL_group_class_t {
    void *(*create)(void *obj, H5VL_loc_params_t loc_params, 
        const char *name, hid_t gcpl_id, hid_t gapl_id, hid_t dxpl_id, 
        void **req);

    void *(*open)(void *obj, H5VL_loc_params_t loc_params, 
        const char*name, hid_t gapl_id, hid_t dxpl_id, void **req);

    herr_t (*get)(void *obj, H5VL_group_get_t get_type, hid_t dxpl_id, 
        void **req, va_list arguments);

    herr_t (*close) (void *grp, hid_t dxpl_id, void **req);
} H5VL_group_class_t;
\end{lstlisting}

The {\tt create} callback in the group class should create a group
object in the container of the location object and returns a pointer
to the group structure containing information to access the group in
future calls.

\textbf{Signature:}
\begin{lstlisting}
    void *(*create)(void *obj, H5VL_loc_params_t loc_params, 
        const char *name, hid_t gcpl_id, hid_t gapl_id, hid_t dxpl_id, 
        void **req);
\end{lstlisting}

\textbf{Arguments:}\\
\begin{tabular}{l p{10cm}}
  {\tt obj} & (IN): Pointer to an object where the group needs
  to be created or where the look-up of the target object needs to
  start.\\
  {\tt loc\_params} & (IN): The location parameters as explained in
  section~\ref{sec:map}. The type can be only {\tt
    H5VL\_OBJECT\_BY\_SELF} in this callback. \\
  {\tt name} & (IN): The name of the group to be created.\\
  {\tt dcpl\_id} & (IN): The group creation property list. It contains
  all the group creation properties in addition to the link creation
  property list of the create operation (an {\tt hid\_t}) that can be
  retrieved with the property {\tt H5VL\_GRP\_LCPL\_ID}.\\
  {\tt gapl\_id} & (IN): The group access property list.\\
  {\tt dxpl\_id} & (IN): The data transfer property list.\\
  {\tt req} & (IN/OUT): A pointer to the asynchronous request of the
  operation created by the plugin.\\
\end{tabular}

The {\tt open} callback in the group class should open a group object
in the container of the location object and returns a pointer to the
group structure containing information to access the group in future
calls.

\textbf{Signature:}
\begin{lstlisting}
    void *(*open)(void *obj, H5VL_loc_params_t loc_params, 
        const char*name, hid_t gapl_id, hid_t dxpl_id, void **req);
\end{lstlisting}

\textbf{Arguments:}\\
\begin{tabular}{l p{10cm}}
  {\tt obj} & (IN): Pointer to an object where the group needs to be
  opened or where the look-up of the target object needs to start.\\
  {\tt loc\_params} & (IN): The location parameters as explained in
  section~\ref{sec:map}. The type can be only {\tt
    H5VL\_OBJECT\_BY\_SELF} in this callback. \\
  {\tt name} & (IN): The name of the group to be opened.\\
  {\tt dapl\_id} & (IN): The group access property list.\\
  {\tt dxpl\_id} & (IN): The data transfer property list.\\
  {\tt req} & (IN/OUT): A pointer to the asynchronous request of the
  operation created by the plugin.\\
\end{tabular}

The {\tt get} callback in the group class should retrieve information
about the group as specified in the {\tt get\_type} parameter.It
returns an {\tt herr\_t} indicating success or failure.

\textbf{Signature:}
\begin{lstlisting}
    herr_t (*get)(void *obj, H5VL_group_get_t get_type, hid_t dxpl_id, 
        void **req, va_list arguments);
\end{lstlisting}

The {\tt get\_type} argument is an {\tt enum}:
\begin{lstlisting}
/* types for all group get API routines */
typedef enum H5VL_group_get_t {
    H5VL_GROUP_GET_GCPL = 0,  /*group creation property list */
    H5VL_GROUP_GET_INFO       /*group info                   */
} H5VL_group_get_t;
\end{lstlisting}

\textbf{Arguments:}\\
\begin{tabular}{l p{10cm}}
  {\tt obj} & (IN): The group object where information needs to be
  retrieved from.\\
  {\tt get\_type} & (IN): The type of the information to retrieve.\\
  {\tt dxpl\_id} & (IN): The data transfer property list.\\
  {\tt req} & (IN/OUT): A pointer to the asynchronous request of the
  operation created by the plugin.\\
  {\tt arguments} & (IN/OUT): argument list containing parameters and
  output pointers for the get operation. \\
\end{tabular}

The {\tt arguments} argument contains a variable list of arguments
depending on the {\tt get\_type} parameter. The following list shows
the argument list, in-order, for each type:

\begin{itemize}
\item {\tt H5VL\_GROUP\_GET\_GCPL}, to retrieve the group creation
  property list of the group specified in {\tt obj}:
  \begin{enumerate}
  \item {\tt hid\_t *ret\_id} (OUT): a pointer to an identifier of the
    group creation property list.
  \end{enumerate}

\item {\tt H5VL\_GROUP\_GET\_INFO}, to retrieve the attribute info:
  \begin{enumerate}
  \item {\tt H5VL\_loc\_params\_t loc\_params} (IN): The location parameters
    explained in section~\ref{sec:map}. 
  \item {\tt H5G\_info\_t *ginfo} (OUT): info structure to fill the
    group info in.
  \end{enumerate}
\end{itemize}

The {\tt close} callback in the group class should terminate access to
the group object and free all resources it was consuming, and returns
an {\tt herr\_t} indicating success or failure.

\textbf{Signature:}
\begin{lstlisting}
    herr_t (*close)(void *group, hid_t dxpl_id, void **req);
\end{lstlisting}

\textbf{Arguments:}\\
\begin{tabular}{l p{10cm}}
  {\tt group} & (IN): Pointer to the group object.\\
  {\tt dxpl\_id} & (IN): The data transfer property list.\\
  {\tt req} & (IN/OUT): A pointer to the asynchronous request of the
  operation created by the plugin.\\
\end{tabular}

\subsection{The Link Function Callbacks}
\begin{lstlisting}
typedef struct H5VL_link_class_t {
    herr_t (*create)(H5VL_link_create_type_t create_type, void *obj,
        H5VL_loc_params_t loc_params, hid_t lcpl_id, 
        hid_t lapl_id, hid_t dxpl_id, void **req);

    herr_t (*move)(void *src_obj, H5VL_loc_params_t loc_params1,
        void *dst_obj, H5VL_loc_params_t loc_params2,
        hbool_t copy_flag, hid_t lcpl, hid_t lapl, 
        hid_t dxpl_id, void **req);

    herr_t (*iterate)(void *obj, H5VL_loc_params_t loc_params, 
        hbool_t recursive, H5_index_t idx_type, H5_iter_order_t order, 
        hsize_t *idx, H5L_iterate_t op, void *op_data, hid_t dxpl_id, 
        void **req);

    herr_t (*get)(void *obj, H5VL_loc_params_t loc_params, 
        H5VL_link_get_t get_type, hid_t dxpl_id, void **req, 
        va_list arguments);

    herr_t (*remove)(void *obj, H5VL_loc_params_t loc_params, 
        hid_t dxpl_id, void **req);
} H5VL_link_class_t;
\end{lstlisting}

\subsection{The Object Function Callbacks}

\begin{lstlisting}
typedef struct H5VL_object_class_t {
    void *(*open)(void *obj, H5VL_loc_params_t loc_params, 
        H5I_type_t *opened_type, hid_t dxpl_id, void **req);

    herr_t (*copy)(void *src_obj, H5VL_loc_params_t loc_params1, 
        const char *src_name, void *dst_obj, 
        H5VL_loc_params_t loc_params2, const char *dst_name,
        hid_t ocpypl_id, hid_t lcpl_id, hid_t dxpl_id, void **req);

    herr_t (*visit)(void *obj, H5VL_loc_params_t loc_params, 
        H5_index_t idx_type, H5_iter_order_t order, 
        H5O_iterate_t op, void *op_data, hid_t dxpl_id, void **req);

    herr_t (*get)(void *obj, H5VL_loc_params_t loc_params, 
        H5VL_object_get_t get_type, hid_t dxpl_id, 
        void **req, va_list arguments);

    herr_t (*misc)(void *obj, H5VL_loc_params_t loc_params, 
        H5VL_object_misc_t misc_type, hid_t dxpl_id, 
        void **req, va_list arguments);

    herr_t (*optional)(void *obj, H5VL_loc_params_t loc_params, 
        H5VL_object_optional_t op_type, hid_t dxpl_id, 
        void **req, va_list arguments);

    herr_t (*close) (void *obj, H5VL_loc_params_t loc_params, 
        hid_t dxpl_id, void **req);
} H5VL_object_class_t;
\end{lstlisting}

\subsection{The Asynchronous Function Callbacks}

\begin{lstlisting}
typedef struct H5VL_async_class_t {
    herr_t (*cancel)(void **, H5ES_status_t *);

    herr_t (*test)  (void **, H5ES_status_t *);

    herr_t (*wait)  (void **, H5ES_status_t *);
} H5VL_async_class_t;
\end{lstlisting}

%%% Local Variables: 
%%% mode: latex
%%% TeX-master: t
%%% End: 
