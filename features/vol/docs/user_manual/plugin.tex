\section{Creating a VOL Plugin}
Each VOL plugin should be of type {\tt H5VL\_class\_t} that is defined
as a {\tt struct} of:

\begin{lstlisting}
/* Class information for each VOL driver */
typedef struct H5VL_class_t {
    H5VL_class_value_t value;
    const char *name;
    herr_t  (*initialize)(void);
    herr_t  (*terminate)(void);
    size_t  info_size;
    void *  (*fapl_copy)(const void *info);
    herr_t  (*fapl_free)(void *info);
    H5VL_attr_class_t          attr_cls;
    H5VL_datatype_class_t      datatype_cls;
    H5VL_dataset_class_t       dataset_cls;
    H5VL_file_class_t          file_cls;
    H5VL_group_class_t         group_cls;
    H5VL_link_class_t          link_cls;
    H5VL_object_class_t        object_cls;
    H5VL_async_class_t         async_cls;
} H5VL_class_t;
\end{lstlisting}

The {\tt value} field is an integer enum identifier that should be
greater than 128 for external plugins and smaller than 128 for
internal plugins. This plugin identifier is used to select the VOL
plugin to be used when creating/accessing the HDF5 container in the
application. Setting it in the VOL structure is required.

The {\tt name} field is a string that identifies the VOL plugin
name. Setting it is not required.

The {\tt initialize} field is a function pointer - MSC not used now!.

The {\tt terminate} field is a function pointer - MSC not used now!.

The {\tt info\_size} field indicates the size required to store the
info data that the plugin needs. That info data is passed when the
plugin is selected for usage with the file access property list (fapl)
function. It might be that the plugin defined does not require any
information from the user, which means the size in this field will be
zero. More information about the info data and the fapl selection
routines follow later.

The {\tt fapl\_copy} field is a function pointer that is called when
the plugin is selected with the fapl function. It allows the plugin to
make a copy if the info data since the user might free it when closing
the fapl. It is required if there is info data needed by the plugin.

The {\tt fapl\_free} field is a function pointer that is called to
free the info data when the fapl close routine is called. It is
required if there is info data needed by the plugin.

The rest of the fields in the {\tt H5VL\_class\_t} struct are
``subclasses'' that define all the object VOL function callbacks that
are mapped to from the HDF5 API layer and will be detailed in the
following sub-sections.

\subsection{Mapping the API to the Callbacks}
In order to keep the number of the VOL object classes and callbacks
concise and readable, it was decided to not have a one-to-one mapping
between API operation and callbacks.

The HDF5 library provides several routines to access an object in the
container. For example to open an attribute on a group object, the
user could use {\tt H5Aopen()} and pass the group identifier directly
where the attribute needs to be opened. Alternatively, the user could
use {\tt H5Aopen\_by\_name()} or {\tt H5Aopen\_by\_idx()} to open the
attribute, which provides a more flexible way of locating the
attribute, whether by a starting object location and a path or an
index type and traversal order. All those types of accesses usually
map to one VOL callback with a parameter that indicates the access
type. In the example of opening an attribute, the three API open
routine will map to the same VOL open callback but with a different
location parameter. The same applies to all types of routines that
have multiple types of accesses.  The location parameter is a
structure defined as follows:

\begin{lstlisting}
/* 
 * Structure to hold parameters for object locations.
 * either: BY_ID, BY_NAME, BY_IDX, BY_ADDR, BY_REF 
 */

typedef struct H5VL_loc_params_t {
    H5I_type_t obj_type; /* The object type of the location object */
    H5VL_loc_type_t type; /* The location type */
    union { /* parameters of the location */
        struct H5VL_loc_by_name loc_by_name;
        struct H5VL_loc_by_idx  loc_by_idx;
        struct H5VL_loc_by_addr loc_by_addr;
        struct H5VL_loc_by_ref  loc_by_ref;
    }loc_data;
} H5VL_loc_params_t

/* 
 * Types for different ways that objects are located in an 
 * HDF5 container.
 */
typedef enum H5VL_loc_type_t {
    /* starting location is the target object*/
    H5VL_OBJECT_BY_SELF = 0, 

    /* location defined by object and path in H5VL_loc_by_name */
    H5VL_OBJECT_BY_NAME, 

    /* location defined by object, path, and index in H5VL_loc_by_idx */
    H5VL_OBJECT_BY_IDX,

    /* location defined by physical address in H5VL_loc_by_addr */
    H5VL_OBJECT_BY_ADDR,

    /* NOT USED */
    H5VL_OBJECT_BY_REF
} H5VL_loc_type_t;

struct H5VL_loc_by_name {
    const char *name; /* The path relative to the starting location */
    hid_t plist_id; /* The link access property list */
};

struct H5VL_loc_by_idx {
    const char *name; /* The path relative to the starting location */
    H5_index_t idx_type; /* Type of index */
    H5_iter_order_t order; /* Index traversal order */
    hsize_t n; /* position in index */
    hid_t plist_id; /* The link access property list */
};

struct H5VL_loc_by_addr {
    haddr_t addr; /* physical address of location */
};

/* Not used for now */
struct H5VL_loc_by_ref {
    H5R_type_t ref_type;
    const void *_ref;
    hid_t plist_id;
};
\end{lstlisting}

Another large set of operations that would make a one-to-one mapping
difficult are the {\tt Get} operations that retrieve something from an
object; for example a property list or a datatype of a dataset,
etc... To handle that, each class of objects has a general get
callback with a {\tt get\_type} and a {\tt va\_list} argument to handle
the multiple get operations. More information about types and the
arguments for each type will be detailed in the corresponding class
arguments.

Finally there are a set of functions for the file and general object
(H5O) classes that are not widely used or interesting enough for
plugin developers to implement. Those routines are mapped to a {\tt
  misc} callback in their respective class.

\subsection{The Attribute Function Callbacks}
The attribute API routines (H5A) allow HDF5 users to create and manage
HDF5 attributes. All the H5A API routines that modify the HDF5
container map to one of the attribute callback routines in this
class that the plugin needs to implement:

\begin{lstlisting}
typedef struct H5VL_attr_class_t {
    void *(*create)(void *obj, H5VL_loc_params_t loc_params, 
        const char *attr_name, hid_t acpl_id, hid_t aapl_id, 
        hid_t dxpl_id, void **req);

    void *(*open)(void *obj, H5VL_loc_params_t loc_params, 
        const char *attr_name, hid_t aapl_id, hid_t dxpl_id, void **req);

    herr_t (*read)(void *attr, hid_t mem_type_id, void *buf, 
        hid_t dxpl_id, void **req);

    herr_t (*write)(void *attr, hid_t mem_type_id, const void *buf, 
        hid_t dxpl_id, void **req);

    herr_t (*iterate)(void *obj, H5VL_loc_params_t loc_params,
        H5_index_t idx_type, H5_iter_order_t order, hsize_t *n, 
        H5A_operator2_t  op, void *op_data, hid_t dxpl_id, void **req);

    herr_t (*get)(void *attr, H5VL_attr_get_t get_type, hid_t dxpl_id, 
        void **req, va_list arguments);

    herr_t (*remove)(void *obj, H5VL_loc_params_t loc_params, 
        const char *attr_name, hid_t dxpl_id, void **req);

    herr_t (*close)(void *attr, hid_t dxpl_id, void **req);
} H5VL_attr_class_t;
\end{lstlisting}

The create callback passes to the plugin the following arguments:

\begin{tabular}{l p{10cm}}
  {\tt void *obj} & Pointer to an object where the attribute needs
  to be created or where the look-up of the target object needs to
  start. \\
  {\tt H5VL\_loc\_params\_t loc\_params} & The location parameters \\
  {\tt } & \\
  {\tt } & \\
  {\tt } & \\
  {\tt } & \\
\end{tabular}

\subsection{The Named Datatype Function Callbacks}

\begin{lstlisting}
typedef struct H5VL_datatype_class_t {
    void *(*commit)(void *obj, H5VL_loc_params_t loc_params, 
        const char *name, hid_t type_id, hid_t lcpl_id, hid_t tcpl_id, 
        hid_t tapl_id, hid_t dxpl_id, void **req);

    void *(*open) (void *obj, H5VL_loc_params_t loc_params, 
        const char * name, hid_t tapl_id, hid_t dxpl_id, void **req);

    ssize_t (*get_binary)(void *obj, unsigned char *buf, size_t size, 
        hid_t dxpl_id, void **req);

    herr_t (*get) (void *obj, H5VL_datatype_get_t get_type, 
        hid_t dxpl_id, void **req, va_list arguments);

    herr_t (*close) (void *dt, hid_t dxpl_id, void **req);
} H5VL_datatype_class_t;
\end{lstlisting}

\subsection{The Dataset Function Callbacks}

\begin{lstlisting}
typedef struct H5VL_dataset_class_t {
    void *(*create)(void *obj, H5VL_loc_params_t loc_params, 
        const char *name, hid_t dcpl_id, hid_t dapl_id, 
        hid_t dxpl_id, void **req);

    void *(*open)(void *obj, H5VL_loc_params_t loc_params, 
        const char *name, hid_t dapl_id, hid_t dxpl_id, void **req);

    herr_t (*read)(void *dset, hid_t mem_type_id, hid_t mem_space_id, 
        hid_t file_space_id, hid_t xfer_plist_id, void *buf, void **req);

    herr_t (*write)(void *dset, hid_t mem_type_id, hid_t mem_space_id, 
        hid_t file_space_id, hid_t xfer_plist_id, 
        const void * buf, void **req);

    herr_t (*set_extent)(void *dset, const hsize_t size[], 
        hid_t dxpl_id, void **req);

    herr_t (*get)(void *dset, H5VL_dataset_get_t get_type, 
        hid_t dxpl_id, void **req, va_list arguments);

    herr_t (*close) (void *dset, hid_t dxpl_id, void **req);
} H5VL_dataset_class_t;
\end{lstlisting}

\subsection{The File Function Callbacks}

\begin{lstlisting}
typedef struct H5VL_file_class_t {
    void *(*create)(const char *name, unsigned flags, hid_t fcpl_id,
        hid_t fapl_id, hid_t dxpl_id, void **req);

    void *(*open)(const char *name, unsigned flags, hid_t fapl_id, 
        hid_t dxpl_id, void **req);

    herr_t (*flush)(void *obj, H5VL_loc_params_t loc_params, 
        H5F_scope_t scope, hid_t dxpl_id, void **req);

    herr_t (*get)(void *file, H5VL_file_get_t get_type, hid_t dxpl_id, 
        void **req, va_list arguments);

    herr_t (*misc)(void *file, H5VL_file_misc_t misc_type, 
        hid_t dxpl_id, void **req, va_list arguments);

    herr_t (*optional)(void *file, H5VL_file_optional_t op_type, 
        hid_t dxpl_id, void **req, va_list arguments);

    herr_t (*close) (void *file, hid_t dxpl_id, void **req);
} H5VL_file_class_t;
\end{lstlisting}

\subsection{The Group Function Callbacks}

\begin{lstlisting}
typedef struct H5VL_group_class_t {
    void *(*create)(void *obj, H5VL_loc_params_t loc_params, 
        const char *name, hid_t gcpl_id, hid_t gapl_id, hid_t dxpl_id, 
        void **req);

    void *(*open)(void *obj, H5VL_loc_params_t loc_params, 
        const char*name, hid_t gapl_id, hid_t dxpl_id, void **req);

    herr_t (*get)(void *obj, H5VL_group_get_t get_type, hid_t dxpl_id, 
        void **req, va_list arguments);

    herr_t (*close) (void *grp, hid_t dxpl_id, void **req);
} H5VL_group_class_t;
\end{lstlisting}

\subsection{The Link Function Callbacks}
\begin{lstlisting}
typedef struct H5VL_link_class_t {
    herr_t (*create)(H5VL_link_create_type_t create_type, void *obj,
        H5VL_loc_params_t loc_params, hid_t lcpl_id, 
        hid_t lapl_id, hid_t dxpl_id, void **req);

    herr_t (*move)(void *src_obj, H5VL_loc_params_t loc_params1,
        void *dst_obj, H5VL_loc_params_t loc_params2,
        hbool_t copy_flag, hid_t lcpl, hid_t lapl, 
        hid_t dxpl_id, void **req);

    herr_t (*iterate)(void *obj, H5VL_loc_params_t loc_params, 
        hbool_t recursive, H5_index_t idx_type, H5_iter_order_t order, 
        hsize_t *idx, H5L_iterate_t op, void *op_data, hid_t dxpl_id, 
        void **req);

    herr_t (*get)(void *obj, H5VL_loc_params_t loc_params, 
        H5VL_link_get_t get_type, hid_t dxpl_id, void **req, 
        va_list arguments);

    herr_t (*remove)(void *obj, H5VL_loc_params_t loc_params, 
        hid_t dxpl_id, void **req);
} H5VL_link_class_t;
\end{lstlisting}

\subsection{The Object Function Callbacks}

\begin{lstlisting}
typedef struct H5VL_object_class_t {
    void *(*open)(void *obj, H5VL_loc_params_t loc_params, 
        H5I_type_t *opened_type, hid_t dxpl_id, void **req);

    herr_t (*copy)(void *src_obj, H5VL_loc_params_t loc_params1, 
        const char *src_name, void *dst_obj, 
        H5VL_loc_params_t loc_params2, const char *dst_name,
        hid_t ocpypl_id, hid_t lcpl_id, hid_t dxpl_id, void **req);

    herr_t (*visit)(void *obj, H5VL_loc_params_t loc_params, 
        H5_index_t idx_type, H5_iter_order_t order, 
        H5O_iterate_t op, void *op_data, hid_t dxpl_id, void **req);

    herr_t (*get)(void *obj, H5VL_loc_params_t loc_params, 
        H5VL_object_get_t get_type, hid_t dxpl_id, 
        void **req, va_list arguments);

    herr_t (*misc)(void *obj, H5VL_loc_params_t loc_params, 
        H5VL_object_misc_t misc_type, hid_t dxpl_id, 
        void **req, va_list arguments);

    herr_t (*optional)(void *obj, H5VL_loc_params_t loc_params, 
        H5VL_object_optional_t op_type, hid_t dxpl_id, 
        void **req, va_list arguments);

    herr_t (*close) (void *obj, H5VL_loc_params_t loc_params, 
        hid_t dxpl_id, void **req);
} H5VL_object_class_t;
\end{lstlisting}

\subsection{The Asynchronous Function Callbacks}

\begin{lstlisting}
typedef struct H5VL_async_class_t {
    herr_t (*cancel)(void **, H5ES_status_t *);

    herr_t (*test)  (void **, H5ES_status_t *);

    herr_t (*wait)  (void **, H5ES_status_t *);
} H5VL_async_class_t;
\end{lstlisting}

%%% Local Variables: 
%%% mode: latex
%%% TeX-master: t
%%% End: 
